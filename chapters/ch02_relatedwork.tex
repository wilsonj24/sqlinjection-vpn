
\chapter{Related Work}
\label{ch:relatedwork}

\section{SQL Injection Techniques}
\label{sec:sql injection techniques}
The way I conducted my study and performed my injection attack was by first having to fully understand what it was I was supposed to do, the best way to do that was to study works and attacks that had already been done that are related to mine. The first article that I ran into is "  Vulnerability assessment \& penetration testing as a cyber defense technology." This article relates to my work because it discusses and aims at penetration testing. The article first goes on to talk about exactly what a vulnerability is. A vulnerability is a weakness in the application which can be an implementation bug or a design flaw that allows an attacker to cause harm to the user of the application and get extra privilege \cite{goel2015vulnerability}.
The article then goes on to talk about vulnerabilities and penetration testing. The main focus of this article is to explain that penetration testing and vulnerability assessments are good techniques to use as a way of cyber defense technology. After the paper has discussed the terminology of what it is they are doing, the paper then goes on to talk about the various techniques that can be used as well as tools that can be used. The paper then concludes by stating that the whole goal of the paper was to inform people about penetration testing and it is used that it has as a cyber defense system. It states that using these tools will allow researchers to develop stronger security solutions.

This study "Advanced SQL injection in SQL server applications" gives it is research on SQL injection in applications. The abstract of this document, explains what the article aims to do. The article talks about the most common 'SQL injection' technique as it applies to the popular Microsoft Internet Information Server. It discusses the various ways that SQL can be injected into the application and addresses some of the issues that are related to this kind of attack \cite{anley2002advanced}.
The introduction of the article, explains and talks about what SQL is and how it relates to databases. SQL injection occurs when an attacker can insert a series of SQL statements into a 'query' by manipulating data input into an application. A typical SQL statement looks like this:
'select id, forename, surname from authors'. In this situation, an attacker can simply append SQL statements at the end of the numeric input. In other SQL dialects, various delimiters are used; in the Microsoft Jet DBMS engine, for example, dates can be delimited with the pound character. The article gives an example code of possible situations and what you might see while you might attempt to perform a SQL injection attack. After the article goes on to talk about possible code you might see with some examples, it then goes into very explicit detail about databases and servers. SQL Server provides a mechanism to allow servers to be 'linked' - that is, to allow a query on one database server to manipulate data on another. The concluding remark about this article simply talks about the importance of why you need to 'lock down the SQL server and gives a list on how to build a SQL server. Overall, this article relates to my work as the goal of this project was to perform a SQL injection attack and this article gives numerous examples and scenarios of possible things that could've happened to me. This article goes into a lot of detail about the steps that you would need to take as well as giving you example query statements of what you might type.



\section{Vulnerability Tools}
\label{sec:Vulnerability Tools}

There are many tools used to test vulnerabilities and this article "Testing and comparing web vulnerability scanning tools for SQL injection and XSS attacks" goes into depth explaining how they can be compared against attacks. This article first talks about how web applications are usually made with a time constraint and because of that, they are usually vulnerable to attacks because of a lack of security measures. The whole point of this article is to discuss the available scanning tools that are out there for SQL injection so that one may not necessarily worry too much about that. Web applications are extremely popular today. Nearly all information systems and business applications (e-commerce, banking, transportation, webmail, blogs, etc) are now built as web-based database applications. They are so exposed to attacks that any existing security vulnerability will most probably be uncovered and exploited, which may have a highly negative impact on users. Automatic web vulnerability scanners are often used by web application developers and system administrators to test web applications against vulnerabilities. Therefore, trusting the results of web vulnerability scanners is essential \cite{fonseca2007testing}.
The article then goes on to talk about security measures and the detection of vulnerabilities within web applications. There are two main approaches to test web applications for vulnerabilities: “white box” and “black box”. To simply explain what each of these techniques is, white box is essentially a technique used when the attacker has an understanding of the network that they are trying to get into whereas black box is the opposite, the attacker doesn't have much knowledge when it comes to the knowledge of the network they are trying to get into. This article had a case study that they did. For the evaluation experiments, we used LAMP (Linux, Apache, Mysql, and PHP) web applications. The server runs Linux and the web server is Apache. This server hosts a PHP-developed web application using a Mysql database. The authors had used three commercial web application vulnerability scanners, that we named Scanner 1, Scanner 2, and Scanner 3. They decided to keep the brand and the versions of the web vulnerability scanners anonymous to assure neutrality and because commercial licenses do not allow in general the publication of tool evaluation results. When conducting the test, they put a table of their finding and found that SQL injection had a higher percentage in all 3 scanner techniques. In the conclusion of this article, they talk about their findings and how to use more web applications so that they can understand the relationship between software faults and vulnerabilities. This work relates to mine because as someone interested in SQL injection I also understand how it may affect a web application and which types of attacks might work.



\section{Intrusion}
\label{sec:Intrustion}

As we have seen there are lots of studies out there about SQL injection and this article "Application Layer Intrusion Detection for SQL Injection" talks about how it can be used with intrusion detection. This article first starts by talking about the motivation of their work. Web applications are becoming increasingly commonplace and accessible. Often the developers of these programs are focused on getting a working application under time pressure and may not implement the best security practices \cite{rietta2006application}.
It then goes on to talk about SQL injection and the different techniques that are associated with that type of injection attack. SQL injection is a technique often used to exploit database systems through vulnerable web applications. The technique allows the attacker to not only steal the entire contents of relational databases but also, in many cases, to make arbitrary changes to both the database schema and the contents. This article aims to examine the threat from SQL injection attacks, the reasons traditional database access control is not sufficient to stop them, and some of the techniques used to detect them. Moreover, it proposes a model for an anomalous SQL detector that observes the database from the perspective of the database server itself. In the conclusion of this article, it talks about possible plans and what it wants to do. In the conclusion, they state that they want to research additional measures to take that could mitigate SQL injection attack risks. This work relates to mine because just like I said earlier, to fully be able to understand what SQL injection is and how it affects other applications, I believe it was also essential for me to understand what measures are taken that used to defer SQL injection attacks.

\section{SQL and Web Applications}
\label{sec:sql & web applications}
In this article "SQL injection." The first thing this article does is ask you "Are your web applications vulnerable?" After that in the Introduction, it gives a little background on web applications and SQL injection. SQL injection is a technique for exploiting web applications that use client-supplied data in SQL queries but without first stripping potentially harmful characters. Despite being remarkably simple to protect against, there is an astonishing number of production systems connected to the Internet that are vulnerable to this type of attack \cite{spett2002sql}.
The objective of this article is to focus the professional security community on the techniques that can be used to take advantage of a web application that is vulnerable to SQL injection and to make clear the correct mechanisms that should be put in place to protect against SQL injection and input validation problems in general. Just like an article, I discussed earlier, this article goes in-depth about how you might perform an injection attack, the types of things you might see, and then how you may fix these security errors. This article relates to my work simply because it shows how one might perform a SQL injection attack which was very helpful to me.


\section{Evading SQL injection}
\label{sec:Evading SQL injection}

Over the numerous articles out there, this article "SQL injection signatures evasion" explains how to evade SQL injections. In this paper, the introduction first starts by looking at a theoretical hacker and gives a theoretical scenario. After that, the article then goes into talking about how this theoretical person may hack a system and informs us about the possible solutions the developers of that system could've taken to ensure that the hack couldn't take place or at least be a lot more difficult to take place. In recent years, Web application security has become a focal center for security experts. Application attacks are constantly on the rise, posing new risks for the organization. One of the most dangerous and most common attack techniques is SQL Injection, which usually allows the hacker to obtain full access to the organization's Database \cite{maor2004sql}.
The paper further demonstrates why these techniques are just the tip of the iceberg of different evasion techniques, due to the richness of the SQL language. Eventually, the conclusion that the research leads to is that protecting against SQL Injection using only signatures is simply not practical. This article relates to my work because like most of the other articles I have mentioned, it discusses theoretically how a system might be hacked and how there could be some measures taken to ensure that did not happen.

Several studies talk about SQL injection and this article "Defeating SQL injection" specifically talks about guarding against SQL injections. Just like the first article I talked about, "Vulnerability assessment \& penetration testing as a cyber defense technology", this article talks about SQL injection as a means of defending against it. The best strategy for combating SQL injection, which has emerged as the most widespread website security risk, calls for integrating defensive coding practices with both vulnerability detection and runtime attack prevention methods \cite{shar2012defeating}.
This article first talks about the SQL query language and it is common uses. Structured query language is a code injection technique commonly used to attack websites in which the attacker inserts SQL characters or keywords into a SQL statement via unrestricted user input parameters to change the intended query's logic. After the article talks about the SQL query language, it then goes into some coding practices and talks about some of the types of databases you can use that will access a server and how developers make mistakes that often lead to vulnerabilities. SQL is the standard language for accessing database servers, including MySQL, Oracle, and SQL Server. Web programming languages such as Java, ASP.NET, and PHP provide various methods for constructing and executing SQL statements, but, due to a lack of training and development experience, application developers often misuse these methods, resulting in SQL injection vulnerabilities (SQLIVs). The article then goes on to talk about some examples of vulnerabilities that a developer might make. An example of a mistake would be "Absence or misuse of delimiters in query strings". When constructing a query string with inputs, a programmer must use proper delimiters to indicate the input’s data type. The absence or misuse of delimiters could enable SQL injection even in the presence of thorough input validation, escaping, and type checking. After giving an example of a possible vulnerability, the article would then give some example code of what someone might see. This article aims to shed light on possible vulnerabilities that developers make so that they won't make this mistake, thus making it harder for someone to perform a SQL injection attack on their code. Where this article relates to my work is mainly in the conclusion section. In the conclusion, this author goes on to talk about how these defenses against SQL injection have both their strengths and their weaknesses. This was good for me because then I was able to get a broader understanding of vulnerabilities and gave me some ideas about how I might do my attack.

\section{SQL Detection}
\label{sec:SQL Detection}


Many studies have been on SQL injection and this article "Detecting SQL injection vulnerabilities in web services" explains how it affects websites. This article first talks about web services and their components and how they are becoming a strategic component in a wide range of organizations. Also, web services are extremely exposed to attacks. Any existing vulnerability will most probably be uncovered/exploited \cite{antunes2009detecting}.
The article aims to show that the approach that they decide to take is a better one than just using a traditional approach that scanner tools use and they provide their results and their findings in a table. In the conclusion of this article, they explain why the tool that they built is better. Our tool was able to detect vulnerabilities that were not detected by the commercial scanners. This work relates to my research because when doing a SQL injection attack, I don't know the type of security the web application is using so this article helps explain that when performing an attack you cannot assume anything about security.

\section{Cloud Infrastructure}
\label{sec:cloud infrastruture}
With the start of the implementation of my prototype, I had to use a type of technology called cloud computing. In "Secure User Data in Cloud Computing Using Encryption Algorithms" cloud computing is transforming information technology. As information and processes are migrating to the cloud, it is transforming not only where computing is done, but also fundamentally, how it is done \cite{arora2013secure}.
This piece explains how important and fundamental cloud computing is when it comes to solving modern-day problems in academics and a working environment. The article explains how new things such as data security and data ownership have made it difficult for normal approaches. Where the article is important to my work is when it starts talking about security issues and challenges in cloud computing. A big problem that this article talks about is the integrity of the data meaning that the data would not be changed unless authorized. The proposed plan that this article talks about is creating an algorithm that eliminates the concerns regarding data loss, and privacy while accessing the web. This article relates to my work because I am implementing a tool using a cloud computing platform to get rid of some of the data and privacy issues that plague information technology processes.

Just as the previous article mentions, there are lots of security issues that cloud computing tries to address. In "Addressing cloud computing security issues" talks about some of those issues. The recent emergence of cloud computing has drastically altered everyone’s perception of infrastructure architectures, software delivery, and development models \cite{zissis2012addressing}.
This paper sheds light on something that is not new in the computer science world, but it is something that I haven't quite seen talked about. When this article addresses the idea of security, it then talks about trust. For the concept and system to work as expected, there must be mutual trust going on. This overall helps with credibility and reliability. The next important concept that this article talks about that also pertains to me is security identification of threats. The information system must identify threats so proper countermeasures can be taken. This concept and idea relate to my work because I have to make sure I am identifying the important security issue(s) surrounding my web application and database so then I can create a prototype using a cloud computing service and eliminate those threats.

\section{VPN services}
\label{sec:VPN services}
When I started implementing my prototype, I needed to use a VPN. This will help change the IP address of the user which in turn will help with better security. This approach is also very important when accessing public and private cloud computing services. In "An automated implementation of hybrid cloud for performance evaluation of distributed database" talks about the idea of hybrid cloud. A Hybrid cloud is an integration of resources between private and public clouds. It enables users to horizontally scale their on-premises infrastructure up to public clouds to improve performance and cut up-front investment costs. This model of application deployment is called cloud bursting that allows data-intensive applications, especially distributed database systems to have the benefit of both private and public clouds \cite{mansouri2020automated}.
According to the article, IT businesses have a large desire to exploit hybrid cloud services. This brings a potential capability to implement a consistent hybrid cloud through VPN. This work is related to my project because I too plan on using a VPN to help with privacy issues when connecting to a database through the cloud.

With time permitting and my prototype shedding possible security solutions, I needed to use a VPN. In "Design of Border Security Defense System for VPN Network in Power Enterprises" explains the importance of using a VPN. This article talks specifically about using a VPN in power enterprises but the concept is very similar to my work. To ensure the safe and stable operation of the power system, the VPN network border security defense system of power enterprises is designed. The hardware of the system is planned and designed, including cooperative intrusion detection, security audit, cooperative camouflage, collaborative firewall, disaster recovery, and electronic forensics \cite{wenzhen2020design}.
Just like my possible solution with my project. A VPN is very important when it comes to doing work online. It allows for privacy and as the article put it, "camouflage".

Why are VPNs so popular today? well in "Virtual Private Networks'' explains the explicit uses of a VPN and why it is so popular today. Let's first start with what a VPN is. A VPN is a Virtual Private network \cite{venkateswaran2001virtual}.
Well, what does that look like? Let's first characterize a VPN. It is a private network that supports a closed community of authorized users, allowing them to access various network-related services and resources \cite{venkateswaran2001virtual}.
Now let us answer the first question that we had asked. Why are VPNs so popular today? Traditional private networks facilitate connectivity among various network entities through a set of links, comprising the dedicated circuit. VPN services enable remote access to the Intranet at a significantly lower cost, thus enabling support for a mobile workforce. Additionally, the VPN architecture supports a reliable authentication mechanism to provide easy access to the Intranet from anywhere using any available access media including analog modems, ISDN, cable modems, DSL, and wireless. There are multiple types of VPN services. The three primary sources of VPNs are Local Area Networks, Dial-up VPN, and Extranet VPN services. LAN Interconnect VPN services help to interconnect local area networks located at multiple geographic areas over the shared network infrastructure. Typically, this service is used to connect multiple geographic locations of a single company. The Dial-up VPN service supports mobile and telecommuting employees in accessing the company’s Intranet from remote locations. An extranet VPN service combines the architecture of LAN Interconnect VPN services and dial-in VPN services. This infrastructure enables external vendors, suppliers, and customers to access specific areas of the company’s Intranet.
A big aspect of the prototype that I plan on implementing uses cloud computing. According to "Cloud Computing: A Perspective Study'' A computing Cloud is a set of network-enabled services, providing scalable, QoS guaranteed, normally personalized, inexpensive computing infrastructures on demand, which could be accessed in a simple and pervasive way \cite{wang2010cloud}. There are three fundamental aspects of the cloud computing platform. There is Hardware as a Service (Haas), Software as a service (Saas), and Data as a Service(Daas). With Haas, IT automation, and usage metering and pricing, users could buy IT hardware, or even an entire data center, as a pay-as-you-go subscription service. The Haas is flexible, scalable, and manageable to meet your needs \cite{wang2010cloud}. With Saas, Software or an application is hosted as a service and provided to customers across the Internet. This mode eliminates the need to install and run the application on the customer’s local computers. SaaS, therefore, alleviates the customer’s burden of software maintenance and reduces the expense of software purchases by on-demand pricing. An early example of the SaaS is the Application Service Provider. Lastly, with Daas, Data in various formats and from multiple sources could be accessed via services by users on the network. Users could, for example, manipulate the remote data and operate on a local disk or semantically access the data on the Internet. DaaS could also be found at some popular IT services such as Google Docs \cite{wang2010cloud}.


