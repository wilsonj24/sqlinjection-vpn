\chapter{Discussion and Future Work}
\label{ch:conclusion}

\section{Summary of Results}

From the graphs and data that we can see throughout the document. I think it is
safe to say that this type of injection technique is very prominent throughout
the cybersecurity world. Having the ability to bypass a login because of sloppy code
that is written to connect a database to a web application is pretty important. Not
only that but we can see how damaging something like this can be. Not only from the examples
that were given in the introduction but also from the way we walked through the
process of how one can do this. We can see that this type of technique can
work. Along with that, we can see that the proposed prototype that I had decided to
work on also looks very promising. As one can see the user can set up a
database that is in the cloud, and only have it accessible if you have the right credentials. The
way we set it up as one can see is that there are multiple layers of security so you can't
just try to bypass some type of security protocol and expect to automatically get the
important and sensitive data. Being able to only access the database with certain
criteria and showing that the data that is both stored on the website and through the
database cluster through the cloud both show good examples of how this prototype can work.
From a look at the results, I think it is safe to say that the proposed prototype
looks like it could be a promising prototype that mitigates vulnerabilities that are associated
with web applications and databases.

\section{Ethics}

A really big key point of this project was the ethical implications surrounding it. The idea behind this project was to not only show how there are security issues surrounding databases associated with web applications but also talk about why someone may try to use a certain technique to gain access to information they are not supposed to see. Here lie the ethical implications surrounding my very topic. One may have the ability to gain access to very sensitive information but does that mean they should do it? When we take a look at the data of the project each "id" represents another human being. Human beings have rights and a big right that they have is the right to privacy and security. This project should not tell you to go out and to try and "hack" different websites. This project is aimed at showing you the ethical implications that surround hacking. However, there are a few ethical reasons as to why someone may try and use techniques such as the one that has been demonstrated. For example, there are certain jobs out there that do ethical hacking. White Box Testing that was described earlier is an example of ethical hacking. What does this look like though? Well, people who work for consulting firms can be hired to try and hack a companies website. This is to try and find out if their sites have any vulnerabilities that they can find and if they can, they will report back with their findings. This will allow companies to continue to improve their security when it comes to important things like websites and databases. This is just an example to show you that there are ethical reasons for hacking. The sad truth of this story though is hacking is used more for unethical reasons. Stealing data, leaking classified information, and deleting important information are all examples of ways people have used hacking techniques unethically. 

\section{Importance}

This type of work is very important. What I have done in this project, is that I have shown that you can access very private information on a web application very easily. People's whole lives are just an id number in databases that are stored online on companies' websites and this can be a problem when the right security measures aren't being taken. This work is showing that there are security issues that have been found, but nobody has gone back to fix these issues. SQL injection is one of the major issues associated with attacking data on web applications so why hasn't this problem been fixed if it has been around since the early 2002's? These are the types of questions I hope to find answers to as I do more research and learn more about security that revolves around databases. MySQL, for example, is more of an industry-standard database most companies use rather than SQLite so the security around that database is better, but it still is not full proof to safeguard against attacks. Lots of people around the world are putting their trust in major companies such as Amazon, Apple, and Microsoft. Yet, Companies like Facebook and Yahoo have had major data breaches and lost the trust of so many of their users. This project aims to continue the ongoing research and development of the best security methods to ensure people's privacy. If major companies like those continue to lose the trust of their users because of breaches our economy will also plummet. Imagine a world where Amazon has been hacked and no one trusts to use their services anymore. Our way of life in terms of receiving packages would completely change.


\section{Conclusion}

With the completion of this project. There were a lot of big key takeaways, not
only from the technical standpoint of security involved with computer science but
also the ethics that are involved with people giving their information away to companies
and companies storing them in a big database. Everyone should not have to give away
so much of their information to create an account with a company. I purposefully chose
the variable that I did to show that if a major company or corporation had a data
breach. Lots of people's information would just be out and the type of information that
sits in those databases are people's lives. When I look at each id in that database, I see a person. Since that id represents another human-being we have to
be very diligent and precise when we are thinking about putting data in a database
that can be accessed online. To me, there isn't enough security in the world to ensure
that the data in the database is secured. It represents a person's whole life. I believe
a project like this should prompt companies and organizations to re-think the types
of requirements that are necessary to open an account with them.

\section{Future Work}

There is a lot of future work that can be done for this project. The further implementation
of my prototype is something that should continue to be worked on as it will
help uncover other possible security vulnerabilities that are related to data and databases
online. With the start of the implementation of my prototype, however, there were numerous
things that I learned that I think are very helpful when it comes to putting sensitive
data in a database. One important thing is that the cloud computing aspect of this
makes this a lot easier as well. When you put data on a database and you connect it
to a web application, essentially your database is just sitting on the same server
being stagnant. With the cloud computing aspect, your database is not stagnant on a  single server, so you just can choose which servers can access that database. This will complicate things
a little because this messes with the client-host server details which are a bit
different from having a database connected to a web application. Another big detail is
the fact that you can make accessing data in a database using a VPN a PRIORITY. This can be extremely helpful when
accessing sensitive data online because then, no one would know where you are originally
signing in from. You are masking your actual IP address so this will make tracking you
and your location much difficult, not to mention this will also make it much more
difficult for hackers to try and figure out what server your database can be accessed
through. As regarding plans for this type of project though, I think something
that should be looked at is implementing this associated with a web application in
some way. If, somehow, we could create a web application page that will redirect the
user to a cloud computing machine and then from there, try to login with the credentials
to try to access the data in the database that would be a step in the right direction. As well, if the user is not connected to that
specific IP address that is in tangent with the VPN associated with the database cluster
then the user should not be able to see any information and if the user is connected to
the proper IP then the user should be granted access.


